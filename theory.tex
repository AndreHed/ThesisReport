%!TEX root = report.tex

\section{Theory} % (fold)
\label{sec:theory}
This section gives a basic introduction to the Finite Element Method (FEM) for structural mechanics, and describes the different stages of finite element analysis (FEA). It also presents the software programs that are used in the current analysis workflow at Andritz Hydro AB.

\subsection{Finite Element Method for Structural Mechanics} % (fold)
\label{sub:finite_element_method_in_structural_mechanics}
Here follows a short but descriptive text about FEM.
When, where and why did it originate?
Elements
Mathematical formulation
System of equations

During the design of structures it is important to analyse the structure's capability to support the applied loads; structural mechanics focuses on the computation of deformations and stresses, to evaluate the performance of the structure. The physical phenomena that are investigated (stress, strain, elasticity etc.) are modelled by differential equations together with a set of boundary conditions. Generally, there exist no analytical solutions to differential equations that relate to complex structures, therefore, the only way to solve the equations are with numerical methods.

The finite element method is as numerical approach that can be used to approximate the solution of boundary value problems. The method has been used rigorously in different engineering disciplines for several decades, and its main advantage over other numerical methods for solving differential equations is its capability to handle complex geometries. The structure is divided into smaller parts, called elements, which together create a discrete mesh of the geometry.~\cite{ottossen92}

(Maybe add section with mathematical FEM description. degrees of freedom)


\subsubsection{Pre-Processing} % (fold)
\label{ssub:pre_processing}
When a structure has been designed, the CAD model needs to be analysed to determine if the desired performance qualifications are reached. The first step of the analysis is the pre-processing, which is the step that prepares the model for a computation.

% Clean: FEA book p. 181-191
Ideally, the CAD model built by the design team is usable for finite element analysis without having to alter the model. However, this is often not the case, since it often exists features that are problematic from an analysis viewpoint. Therefore, it is necessary to make simplifications to the model in order to create a mesh of the model. Design features that complicate automatic meshing are short edges, sliver faces, small holes, fillets, chamfers etc., and it is necessary to clean the model of these features to enable the automesher to mesh the model. This cleaning process can be both tedious and difficult since it may exist a lot of features, and even if a disadvantageous feature is found it may be difficult to remove it; the cleaning process can therefore be very time consuming. It is also important to mention that the simplifications that is performed should not influence the structural capabilities of the model, this could be difficult to assess, but as long as the simplifications are local and not in areas of interest the structural capabilities should not be affected.~\cite[p.~181--191]{adams99}

% Meshing
When the model is clean, the next step is to mesh the model. Whether the model is meshed automatically or manually and with shell elements or tetrahedrons, the process is an essential step of the analysis. The focus of this thesis is on automeshing with tetrahedrons (for 3D models) and triangles (for 2D models). Automeshing is a simple and efficient technique to create a discrete FE model of the CAD model. Depending on which automesher that is used different settings are available, but the user should generally consider the maximum and minimum elements size, element growth rate and local mesh refinements. The goal is to create a mesh that captures the structural features of the model with as few elements as possible. Even though the automesher could create the mesh in a fast and convenient way, it is no guarantee that the mesh is sufficiently refined. Therefore, considerable thought should be spent on where local mesh refinements are necessary, how small elements are needed and how large elements are excepted. It is also important to mention that it is an absolute necessity that the model is clean, otherwise the automesher will most likely fail to create a mesh.~\cite[p.~251-255]{adams99}

Part of the pre-process is also to define the boundary conditions that influence the model. There exist two groups of boundary conditions: constraints that prohibits the model from moving in the specified spatial degrees of freedom, and loads such as forces, moments and temperature. How the boundary condition are defined and which types that are used are often not straightforward, and an important part of the analysis.~\cite[p.~263]{adams99}

Before the FE model can be solved the material properties of the model needs to be specified.

It is probably obvious that this part of the analysis can be very time consuming, and to obtain a solution it is of paramount importance that the FE model is created correctly, with sufficient details of the original model and that boundary conditions are properly defined.
% subsubsection pre_processing (end)

\subsubsection{Solution} % (fold)
\label{ssub:solution}
When the FE model is created, the next step is to solve the model to obtain the results. This step is mainly executed by the computer, the only effort from the user is setting the correct solver parameters. Which parameters that can be specified is highly dependent on which solver is used. The execution time depends on how large the FE model is (number degrees of freedom) and what solution type is used (non-linear solutions are more demanding).

When the solver is finished it is important to evaluate the results and the solver output to establish if the solution is reasonable and if the results are accurate. To determine if the mesh is sufficiently refined the convergence could be checked, and to check that the boundary conditions are properly defined the resultant forces on the model could be compared with the specified loads.~\cite[p.~303-324]{adams99}
% subsubsection solution (end)

\subsubsection{Post-Processing} % (fold)
\label{ssub:post_processing}
The first step of the post-processing is to visualise the results (displacement, stress, etc.) to determine if the solution is reasonable. If the solution is reasonable the FE model can be considered to be adequate. The next step is to visualise the specific results that are the goal of the analysis. Even if this step is not very technical, it is important to analyse the results to determine if the results can be trusted.
% subsubsection post_processing (end)

\subsubsection{Elements} % (fold)
\label{ssub:elements}
Describe different types of elements.
% subsubsection elements (end)

\subsubsection{Boundary Condidtions} % (fold)
\label{ssub:boundary_condidtions}
Describe different types of bc's.
% subsubsection boundary_condidtions (end)

\subsubsection{Load Cases} % (fold)
\label{ssub:load_cases}
Maybe not necessary
% subsubsection load_cases (end)

\subsubsection{Contacts} % (fold)
\label{ssub:contacts}
Describe non-linearity, meshing technique etc.
% subsubsection contacts (end)

% subsection finite_element_method_in_structural_mechanics (end)

\subsection{Siemens NX} % (fold)
\label{sub:siemens_nx}
Write some general about NX and which features that are used. Also write about Advanced Simulation.

NX\texttrademark{} is a software developed by Siemens PLM Software that supports every aspect of product development (i.e. design, analysis and manufacturing)~\cite{siemensnx}. This section describes a subset of the functionality provided by NX that is relevant for this thesis, and the focus is on the tools to clean the model (called Synchronous Modelling  tools) and the analysis solution (called Advanced Simulation).~\cite[p.~36ff.]{goncharov14}

The synchronous modelling tools that are provided by NX are very efficient and powerful to use during the cleaning process. Synchronous modelling extends the regular parametric modelling approach that is used in CAD with more intuitive and direct modelling tools. The tools that are most commonly used are: \textit{Move Face}, \textit{Delete Face}, \textit{Make Coplanar}, \textit{Pull Face} etc., for a description of these tools see~\cite{goncharov14}. The main advantage of these tools is that the model can be changed without having access to the or making changes to the original history-tree of the commands that created the model.

NX Advanced Simulation provides the entire CAE workflow with pre- and post-processing and solver environment. It is also tightly integrated with the CAD application of NX, enabling a fast and efficient analysis process. The meshing capabilities that Advanced Simulation provides are the usual 3D and 2D automeshers as well as more exotic elements for specific applications. The simulation environment can be integrated with several of the most common solvers such as NX Nastran, Abaqus and Ansys.
% subsection siemens_nx (end)

\subsection{Salomé Platform} % (fold)
\label{sub:salom_platform}
Salomé is a free software platform (distributed under GNU LGPL~\cite{lgpl}) for numerical simulation. Salomé strives to provide a framework where the entire workflow of a numerical simulation (described in sec.~\ref{ssub:pre_processing}-\ref{ssub:post_processing}) can take place. The platform consists of different modules that can be used for pre- and post-processing, which are presented under the same GUI (graphical user interface).~\cite{ribes07} 

\subsubsection{Geometry Module} % (fold)
\label{ssub:geometry_module}
The \textit{Geometry} module provides basic functionality to import CAD models and prepare them for numerical simulation. Common tasks that are carried out in this module is:
\begin{itemize}
 	\item create groups
 	\item partition objects
 	\item whateva
\end{itemize}

A group in Geometry is a collection of geometrical objects (solids, faces, edges or vertices), which is given a common name and can be referenced by that name from other modules. Groups are used for three different purposes: local mesh refinement, boundary conditions and contact surfaces.

To be able to control the automeshing it is useful to define patches on the model where local mesh refinements are necessary to get the solution to converge. A patch is a group of faces or edges that should be better resolved. Any geometrical object of the model that is the subject of a boundary condition should also be defined as a group. If two objects are in contact, and a contact simulation is required, the contact surfaces should be defined as groups.

The partition function connects solid objects and creates a single solid, but the contact face will still remain between the objects. This function is mainly used when a conformal mesh has to be created for a contact simulation.~\cite{salomedoc}
% subsubsection geometry_module (end)

\subsubsection{Mesh Module} % (fold)
\label{ssub:mesh_module}
When a model has been imported to Geometry and and the groups are defined, the \textit{Mesh} module can mesh the model. Mesh provides the most common elements and several different algorithms for mesh generation are available, resulting in a very extensive library of functions.

To mesh an object with the Mesh module first an object from Geometry is selected, then an algorithm is selected and finally an hypothesis is created. The hypothesis represent the parameters that the algorithm use to create the mesh, depending on the algorithm different parameter values are specified in the hypothesis. Since there exist several algorithms in Mesh, this section only describes the \textit{NETGEN} algorithm and the \textit{Projection} algorithm.

The NETGEN algorithm~\cite{netgen} is an automesher that can be used for creating 3D tetrahedral meshes and 2D triangular and quadrilateral meshes. The hypothesis for the NETGEN algorithm specifies
\begin{itemize}
	\item the maximum and minimum element size, 
	\item if second order elements should be used,
	\item growth rate,
	\item the possibility to specify the number of elements based on surface curvature, and
	\item local refinements on specified groups. 
\end{itemize}

The projection algorithm creates a mesh of an object by projecting the mesh of another object. This algorithm is especially used when conformal meshes are desirable.

The mesh module supports, just as the Geometry module, the creation of groups. This feature is specifically useful when a specific part of the mesh needs to be referenced or exported.
% subsubsection mesh_module (end)

\subsubsection{ParaVis Module} % (fold)
\label{ssub:paravis_module}
The ParaVis module is based on ParaView an open source platform for data analysis and visualisation. Suffice to say that all the functionality needed to visualise the results in a structural analysis is provided.
% subsubsection paravis_module (end)

\subsubsection{Application Programming Interface (API)} % (fold)
\label{ssub:application_programming_interface_}
A very advantageous feature of Salomé and the modules that have been described is that the functionality provided by the GUI also is available through an API, enabling the user to write scripts that execute functions automatically.

An API is a framework of functions for building software applications. Using the API a developer can efficiently create software that perform specific tasks by using the functionality provided by the API. The documentation of the API describes which functions that are available and what each functions does.

Salomé's API is based on the Python programming language and the Salomé GUI includes a Python console, by which the user can execute scripts that utilise the API. There exist a specific API for each module and these can work together within the same script, therefore it is possible to write a script that first import a model to Geometry where groups are created, then the script can select an algorithm and create a hypothesis, and finally the script can update the GUI.

\begin{figure}[t]
	\begin{center}
		\begin{tikzpicture}
			\node[rect] (salome)	{Salomé}; 
		\end{tikzpicture}
	\end{center}
	\caption{Some schematic view of the Salomé platform}
	\label{fig:salome}
\end{figure}

% subsubsection application_programming_interface_ (end)

% subsection salom_platform (end)

\subsection{Code\_Aster} % (fold)
\label{sub:code_aster}
Code\_Aster~\cite{codeaster} is an open source FEM software for structural analysis which is developed by EDF (Électricité de France) and distributed under GPL (General Public License)~\cite{gpl}. The software is driven by an input file with commands that that describe the simulation (a COMM file), and a text file describing the mesh (usually a MAIL file). The information the MAIL file needs to contain is the mesh data, that is the nodes and the elements of the mesh, but the file also needs to describe the patches on which the boundary conditions are applied, these patches can be created in the Mesh module as described in Section~\ref{ssub:mesh_module}. The output is text files describing general information about the simulation (errors, warnings, convergence, etc.) and a file containing the results which can be imported to Salomé and visualised with the ParaVis module. 
% subsection code_aster (end)

% section theory (end)
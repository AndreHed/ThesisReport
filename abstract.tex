%!TEX root = report.tex

The Finite Element Method (FEM) is a technique for finding the approximate solution of differential equations. It is commonly used in structural analysis to evaluate the deformation and internal stresses of a structure that is subject to outer loads. This thesis investigates the Finite Element Analysis (FEA) workflow that is used at Andritz Hydro AB, with the objective to find solutions that make the workflow more effective. The current workflow utilises \textit{Siemens NX} and \textit{Salomé} for pre- and post-processing, and \textit{Code Aster} as the FEM solver. Two different approaches that improve the workflow are presented. The first suggest that the entire FEA workflow is migrated to NX using the built-in FEM package of NX called \textit{Advanced Simulation}. The second approach utilises the Salomé API (Application Programming Interface) to create a customised toolbox (a script containing several functions) that automate several repetitive and cumbersome steps of the workflow, therefore effectively reducing the time that is required by the analyst to perform FEA. Due to the effective results and ease-of-use, the Salomé toolbox is preferred over the license cost and steep learning curve that is related to NX and Advanced Simulation. 

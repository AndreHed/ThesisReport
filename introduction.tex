%!TEX root = report.tex

\chapter{Introduction}
\label{cha:introdution}
% \subsection{Hydro-Electric Power}
Hydro-power is the worlds largest renewable energy source, and hydroelectric power plants have been developed and used since the nineteenth century. During the last decade hydro-power amount to 45 percent of the total electricity production in Sweden~\cite{scb}. Many of the hydro-power plants in Sweden were built several decades ago, and they are therefore in need of refurbishment and modernisation. A simplified description of a conventional hydro power plant is that it consist of a dam, turbine and a generator, where the dam creates a water reservoir that contains potential energy that drives the turbine which in turn generates electricity.

% \subsection{Andritz Hydro}
The Andritz Group provides services mainly for the hydro-power, pulp and paper, and metals industries, with headquarters in Graz, Austria and approximately \num{24500} employees worldwide. Andritz Hydro is a supplier of electromechanical equipment for hydro-power plants, and the Swedish subsidiary Andritz Hydro AB (henceforth referred to as Andritz) has 160 employees and focuses mainly on refurbishment and optimisation of turbines and generators.

% Write about the 'product realization process'
At Andritz the entire product development process is performed, including product design, analysis and manufacturing. The design process, usually referred to as CAD (Computer Aided Design), and the analysis process, usually referred to as CAE (Computer Aided Engineering), are the parts of the product development that is the focus of this thesis.

Turbines and generators are advanced electromechanical equipment, and since hydroelectric power plants need to reliably operate under long terms, the modernisation process is subject to rigorous analysis to ensure long term operation. Therefore, it is important to analyse the structure's capability to support the applied loads; \textit{structural mechanics} focuses on the computation of deformations and stresses, to evaluate the performance of the structure. The physical phenomena that are investigated (stress, strain, deformation, etc.) are modelled by solving differential equations together with a set of boundary conditions. Generally, there exist no analytical solutions to differential equations that relate to complex structures, therefore, the only way to estimate solutions of the equations are with numerical methods.

The finite element method (FEM) is a numerical approach that can be used to approximate the solution of boundary value problems. The method has been used rigorously in different engineering disciplines for several decades, and its main advantage over other numerical methods for solving differential equations is its capability to handle complex geometries. The structure is divided into smaller parts, called elements, which together create a discrete mesh that represent the geometry.~\cite{ottossen92}

% Write about why the workflow is problematic at Andritz
There exist several different software programs for CAD and CAE, at Andritz \textit{Siemens NX} is used for CAD and \textit{Salomé} together with \textit{Code Aster} are used for the finite element analysis (FEA). The analysis process is comprised of three steps: pre-processing, solution and post-processing. The goal of the pre-processing is to from a CAD model develop an FE model containing a mesh, material definitions and boundary conditions. In general, a significant amount of time is spent on the pre-processing and a lot of manual work can be required by the analyst, it is therefore important to make the pre-processing as effective as possible.

At Andritz the analysis workflow is in some aspects cumbersome and convoluted, and a more streamlined workflow is desired. Some of the difficulties occur solely because of that two different software programs are used for CAD and CAE, and other difficulties are generically inherited from the software programs that are used.

The purpose of this thesis is to evaluate the current workflow at Andritz, based on time-efficiency and convenience. Based on what the evaluation manifest, suggestions to improve the workflow will be presented. The suggestions are concerned with both larger strategical improvements, such as evaluating other software programs, and smaller improvements that simplify time-consuming repetitive processes. The main aspects to consider are:
\begin{itemize}
	\item time-efficiency,
	\item simple and convenient processes,
	\item cost-efficiency and
	\item easy and fast to learn.
\end{itemize}

Since the product development process is complex, it is not certain that there exist feasible solutions that improves the workflow. It is also not guaranteed that the software programs provides functionality such that processes can be simplified.


%\subsection{Finite Element Analysis}
%
%\subsection{Finite Element Analysis at ANDRITZ} \label{sec:feaandritz}
%
%When a part is subject to an analysis the calculation engineer start with a part in NX. The part is idealised by removing smaller holes, chamfers and fillets that are non-essential for the analysis. Idealising the part can be done with the synchronous modelling commands that are available in NX. The part is now exported as a \texttt{.step} file that contains the geometrical description of the part.
%
%The \texttt{.step} file is imported into Salome-Meca a software platform for pre- and postprocessing. The idealisation that is done in NX is the first step in the preprocessing of the part. The next step is to create the necessary groups where boundary conditions and loads will be applied. The final step of the preprocess is to mesh the part, the part needs to be divided into a set of elements that describe the original model in accurately. The part is now exported from Salome-Meca, a text file that describes the geometry, the mesh and the defined groups is produced.
%
%The next part of the analyses is to solve the case by first defining the boundary conditions, load cases and physical properties that are needed to create a tractable model. These steps are done with command-line driven open source FEM package called Code Aster. The text file that Salome-Meca produced describes the geometry, mesh and groups that adheres to the \texttt{.unv} format, this format is not preferred by Code Aster, therefore the \texttt{.unv} file is converted to a \texttt{.mail} file. The case setup is described in a \texttt{.comm} file that is read by Code Aster.
%
%The post-processing is performed in Salome-Meca where the results created by Code Aster is imported. The final analysis of the results are visualised as needed.
%
%\begin{figure}
%	\centering
%	\rule{2cm}{2cm}
%	\label{fig:workflow}
%	\caption{A schematic figure of the workflow at Andritz for finite element analyses.}
%\end{figure}
%
%\subsection{Goal}
%The main focus of this thesis is to investigate the possibility to imporove this workflow process, to see if any changes are possible and if they would improve efficiency. There are several perspectives that could be investigated, and the main focus of this thesis will be on:
%\begin{itemize}
%	\item NX provides a utility called \emph{Advanced Simulation} that includes the functionality to both create groups and mesh from NX. If \emph{Advanced Simulation} fulfils the requirements of the current process, it would be possible to eliminate several steps from the workflow that is described in section~\ref{sec:feaandritz} and Figure~\ref{fig:workflow}.
%	\item \emph{Advanced Simulation} also includes the functionality to use some of the eminent solvers Nastran, Abaqus and Ansys. Therefore it is of interest to investigate how this functionality works in NX.
%\end{itemize}
%Eventhough the focus of the investigation is to evaluate the technical functionalities of NX and \emph{Advanced Simulation}, it is important to establish that the current NX license that is used at Andritz does not include the \emph{Advanced Simulation} utility. Therefore, the investigated paths described above would result in an increase of the costs, since the current softwares (Salome-Meca and Code Aster) are open source and free of charge.
%
%The investigation will also take another perspective, which is to try and improve the current workflow, without changing any of the main steps. The current workflow contains several exports and imports between different softwares and a general analysis could include several iterations between the calculation team and the design team. During these iterations, small, but significant changes are made on the part that is under analysis. Due to the convoluted workflow it is difficult to reuse the settings and work done on the previous part, therefore the process has to be repeated even if just a small change has been done to the part. This is particularly evident in the preprocessing step in Salome-Meca, where the defined groups and mesh settings are lost between these iterations. 
%
%Salome-Meca provides a python API (Application Programming Interface) such that the user might create its own scripts to perform tasks that are not provided by Salome-Meca. With this functionality it is possible to investigate if a script can be written such that it can compare the part from the previous iteration with the new part to see if some of the groups and meshes can be copied to the new part. If this is possible then it would eliminate the nuisance of having to recreate the settings between the iterations.
%
%
%





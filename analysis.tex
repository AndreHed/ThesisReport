%!TEX root = report.tex

\chapter{Analysis} % (fold)
\label{cha:analysis}
The previous section described structural analysis and FEA in general and presented different software programs that can be used in the analysis. This section presents how FEA is performed at Andritz based on the investigation made by the author of this thesis. The investigation is the basis of the improvement suggestions that are given in this section.

\section{Analysis Workflow at Andritz} % (fold)
\label{sec:analysis_workflow_at_andritz}
% Product development process: FEA p. 12-13
The structural analysis starts with a CAD model created by the design team. The design engineers at Andritz use Siemens NX to create CAD models and internally in NX these models are called \textit{parts}. The parts are given a reference number to a database where they are stored, and this reference number is passed on to the calculation team when an analysis of the part is necessary. As described in Section~\ref{sub:pre_processing}, the CAD model needs to be cleaned of features that are unnecessary to the analysis. Since the cleaning process can change the model significantly it is not appreciated by the design engineers that the model they created is altered, therefore, a copy of this model is created which can be prepared for analysis. A copy is created by creating a new part in the database that is assigned to the calculation engineer and then using a synchronous modelling tool called \textit{WAVE Geometry Linker} to link the original CAD model.

The cleaning process is done in NX with the synchronous modelling tools that is described in Section~\ref{sub:siemens_nx}. Depending on the level of detail of the model that is provided, the amount of work that is needed to clean the CAD model can range from very little to a significant part of the entire analysis process. 

When the part is ready to be meshed, it is exported from NX. There exists several formats to export a CAD model such that the details of the model are preserved, and at Andritz the STEP (standard for the exchange of product model data) format is used. A STEP file describes a CAD model according to an ISO standard which is used for file exchange~\cite{stepiso}.

The rest of the pre-processing is carried out in the Salomé platform described in Section~\ref{sec:salom_platform}. The STEP file is imported and the Geometry module is activated. As described in Section~\ref{ssub:geometry_module}, groups are created in the Geometry module to define where local mesh refinement is needed, boundary conditions are applied and contact surfaces exists.

After the groups are defined the analyst change to the Mesh module where the model is meshed. The meshing process in Salomé is described in Section~\ref{ssub:mesh_module}. 3D models are usually meshed with tetrahedrons (see Sec.~\ref{sub:elements}) and 2D models are usually meshed with triangles, in both cases the NETGEN algorithm is used.

To solve the model Code Aster is used and, as is described in Section~\ref{sub:code_aster}, Code Aster requires the mesh to be described in a text file where patches, on which boundary conditions are applied, are defined. Therefore, mesh groups are defined in the Mesh module before the mesh is exported.

The final steps of the analysis (solution and post-processing) has not been investigated to the same level of detail, and are therefore summarised just to give a complete description of the analysis workflow.

Given a file describing the mesh and a COMM file that describes the simulation, Code Aster can run the simulation. The COMM file describes the type of simulation, boundary conditions and other parameters depending on the type of simulation. The output of the simulation is the results that can be visualised and analysed with the ParaVis module.

The entire workflow described above is presented in Figure~\ref{fig:andritz_workflow}.

\begin{figure}[t]
	\begin{center}
		\begin{tikzpicture}[node distance=5mm]
			\node[rect] (design)					{Design};
			\node[rect] (clean)	[below=of design] 	{Model clean-up};
			\node[rect] (step)	[below=of clean]	{Export part as \texttt{.step} file};
			\node[rect] (salom)	[below=of step]		{Import to Salomé};
			\node[rect] (group) [below=of salom] 	{Create groups};
			\node[rect] (mesh)	[below=of group]	{Create mesh};
			\node[rect] (mail)	[below=of mesh]		{Export as \texttt{.mail} file};
			\node[rect] (comm)	[below=of mail]		{Write Code Aster input file};
			\node[rect] (solve)	[below=of comm]		{Solve};
			\node[rect] (vis)	[below=of solve]	{Visualise and analyse results};
			\path (design)	edge[->]	(clean)
				  (clean)	edge[->]	(step)
				  (step)	edge[->]	(salom)
				  (salom)	edge[->]	(group)
				  (group)	edge[->]	(mesh)
				  (mesh)	edge[->]	(mail)
				  (mail)	edge[->]	(comm)
				  (comm)	edge[->]	(solve)
				  (solve)	edge[->]	(vis);
			\draw [->] (vis.west) -- ++(-.5,0) |- (design.west);
		\end{tikzpicture}
	\end{center}
	\caption{Current analysis workflow at Andritz.}
	\label{fig:andritz_workflow}
\end{figure}

If the results are feasible the development of the CAD model can continue, and preparations for the manufacturing process can begin. However, if the results show that the model does not fulfil the requirements, the model needs to be redesigned and the analysis repeated on the updated model. This iteration procedure is standard in product development. Since the analysis of the updated model is likely to be very similar to the original model some of the steps described in Figure~\ref{fig:andritz_workflow} are reusable, however, some steps have to be performed again. During the analysis of a redesigned model the following steps cannot be reused:
\begin{itemize}
	\item Model clean-up
	\item Create geometric groups
	\item Create mesh
	\item Create mesh groups
\end{itemize}
Obviously the visualisation and analysis of the results need to be repeated, but that is unavoidable. The four steps mentioned above are all part of the pre-process of the analysis, and they could be a very time consuming part of the analysis, so it would be favourable to find a solution that could reuse the fact that these steps already have been performed on a very similar model. The reason why these steps have to be repeated is because the link between the CAD model and the analysis model is broken during the first step when the CAD model is copied. Therefore, all succeeding steps of the pre-process needs to be repeated.
% subsection analysis_workflow_at_andritz (end)

\section{Using NX Advanced Simulation} % (fold)
\label{sec:using_nx_advanced_simulation}
The previous section described the structural analysis workflow at Andritz and parts of the workflow that are inefficient where highlighted (especially if a new iteration is started). This section describes a possible solution to some of the problems that are discussed. NX Advanced Simulation (described in Sec.~\ref{sub:siemens_nx}) supports the entire FEA process, and since the CAD model is created in NX it can seem like a good fit to analyse the model in the same software.

\subsection{NX Licenses} % (fold)
\label{sub:nx_licenses}
Currently Andritz has one license for the Advanced Simulation application, this license is called \textit{MasterFEM (A026)} and includes pre- and post-processing capabilities and the NX Nastran Basic solver. MasterFEM is a legacy license package that is no longer provided by Siemens, it has been replaced by a license called \textit{NX Advanced Simulation (NX13500)} which has similar capabilities. If Advanced Simulation should be used at Andritz it is necessary to purchase additional licenses and,therefore, the cost must be considered.

Previous NX licenses where purchased from Conex Software, and the price information they have provided is presented in Table~\ref{tab:price_information}. The difference between the two licenses is that Advanced FEM does not include the NX Nastran Basic solver.

Note that these licenses include a variety of features (for a complete description see~\cite{siemensnx}), but the functionality they provide is sufficient for the analyses that are performed at Andritz.

\begin{table}[tb]
	\caption{Price information for NX pre- and post-processing and solver features.}
	\label{tab:price_information}
	\begin{center}
		\begin{tabular}{lcc}
		\midrule
		\midrule
		\textbf{License} & \textbf{Lic. No.} & \textbf{Price} (SEK/license) \\
		\midrule
			NX Advanced Simulation & NX13500 & \num{137900} \\
			NX Advanced FEM & NX12500 & \num{101920} \\
		\midrule
		\midrule
		\end{tabular}
	\end{center}
\end{table}
% subsection nx_licenses (end)

\subsection{Advanced Simulation Workflow} % (fold)
\label{sub:advanced_simulation_workflow}
To perform a FEA in Advanced Simulation the starting point is a \texttt{part} file describing the CAD model. When a new simulation is started three files are created.
\begin{itemize}
	\item The \textit{idealised} \texttt{part} file is associatively connected (optionally) to the original CAD model, and the purpose of this file is to create a simplified model of the original CAD model.
	\item The \texttt{fem} file represents the FE model, where material properties are defined and a mesh is created.
	\item The \texttt{sim} file where boundary conditions and solver settings are specified.
\end{itemize}
These three files together define the simulation and they are tightly connected such that if there is a change in the idealised \texttt{part} the \texttt{fem} and \texttt{sim} file is automatically updated and the mesh and boundary conditions are adapted to the change. Since the idealised \texttt{part} file is associatively connected to the CAD model this feature is very efficient when a new design-analysis iteration is started.

See Appendix~\ref{sec:nx_advanced_simulation_guide} for a detailed description of the workflow in Advanced Simulation. 
% subsection advanced_simulation_workflow (end)

% subsection using_nx_advanced_simulation (end)

\section{Salomé API Script} % (fold)
\label{sec:salom_api_script}
Salomé provides a Python API for every module, which allows the user to write Python scripts to automate processes that are often repeated. As is described in Section~\ref{sec:analysis_workflow_at_andritz}, there are steps in the pre-process of the FEA workflow that can be time consuming to perform manually, and it is especially inefficient if a new design-analysis iteration is started since the steps that already have been done must be repeated. This section presents a solution that efficiently automate some parts of the pre-processing steps and at the same time reduces the amount of work that have to be repeated if a new iteration is started.

The main issue with the workflow in Figure~\ref{fig:andritz_workflow} is that the connection between the CAD model and the FE model is lost when the part is exported from NX and imported to Salomé. All the steps of the pre-processing that is done in Salomé needs to be repeated with the current workflow. Therefore, it would be better to move as many of the steps as possible from Salomé to NX, since these steps would not have to be repeated. Within the current NX license at Andritz (MasterFEM), the step that can be moved is the definition of the groups.

The main idea of the new workflow is to clean the model and define the groups in NX, and when the model is exported the group definitions are included in the STEP file. In Salomé the model is imported and the groups defined in NX are automatically created. The model is meshed and the groups that should be included in the text file describing the mesh are automatically created in the Mesh module.

\subsection{Defining Groups in NX} % (fold)
\label{sub:defining_groups_in_nx}
In NX any geometrical object (solid, face, edge, etc.) or collection of objects can be given a name. The name and which geometrical object it is associated with is included in the STEP file when the model is exported. The import utility in Salomé does not recognise these names as groups when the STEP file is imported, therefore these names needs to be translated to group objects. The Geometry API provides several functions to automate this process, and the solution is a function named \texttt{CreateGroups} which is presented in Appendix~\ref{sec:creategroups}. \texttt{CreateGroups} creates the groups that are defined in NX of the object that is selected in the object browser.
% subsection defining_groups_in_nx (end)

% TODO: I could write more about the internal process of CreateGroups
% The \textit{Explode} function creates sub-shapes of a specified type of an object, for example if a box is exploded to faces all eight faces of the box are created. Using this function the groups defined in NX can be recreated.

\subsection{Creating a Mesh} % (fold)
\label{sub:creating_a_mesh}
The calculation team at Andritz usually use the NETGEN algorithm with similar settings every time a mesh is created. Since the process of creating meshes in the Mesh module can be repetitive, a function has been developed that creates a mesh using predefined settings and automatically adds the groups that should be locally refined. The function is called \texttt{CreateMesh} (see App.~\ref{sec:createmesh}), and works by selecting an object in the Object Browser and then executing the function. If the object that is selected has any group with a name starting with \texttt{FIN}, that group is automatically added to the local sizes.
% subsection creating_a_mesh (end)

\subsection{Creating a Contact Mesh} % (fold)
\label{sub:creating_a_contact_mesh}
As described in Section~\ref{ssub:contact_mesh} it is complicated to create a mesh with matching nodes at contact surfaces. Therefore, two functions have been developed to automate the process of creating a contact mesh: \texttt{PartitionShapes} and \texttt{CreateSubMesh}.

\texttt{PartitionShapes} partitions several objects and identifies contact surfaces based on a naming convention for groups, the groups are placed in a hierarchical tree that is required by the Projection algorithm (see App.~\ref{sec:partitionshapes}). \texttt{CreateSubMesh} creates a sub-mesh using the Projection algorithm to ensure that the contact is meshed with matching nodes.

The naming conventions under which these functions operate are:
\begin{itemize}
	\item The objects that are in contact have names consisting of two upper case characters (eg. \texttt{CY} for \texttt{Cylinder} and \texttt{BO} for \texttt{Box}).
	\item The contact surface is named by the two objects that are in contact, separated by an underscore (eg. \texttt{CY\_BO}).
\end{itemize}
If the objects are named by these conventions, the workflow for creating a contact mesh is:
\begin{enumerate}
	\item Select the objects that are in contact and execute \texttt{PartitionShapes}.
	\item Select the contact group and execute \texttt{CreateSubMesh}.
\end{enumerate}
% subsection creating_a_contact_mesh (end)

\subsection{Workflow with Salomé Script} % (fold)
\label{sub:workflow_with_salom_script}
In this section a new, more efficient, workflow is proposed that utilise the functions that are described in Sections~\ref{sub:defining_groups_in_nx}-\ref{sub:creating_a_contact_mesh}, see Figure~\ref{fig:script_workflow}. There are several advantages of this workflow and the main features are summarised below:
\begin{itemize}
	\item The synchronous tool WAVE Geometry Linker creates a copy of the CAD model that is associative, therefore, any change to the CAD model can update the FE model in NX. If groups are defined in NX, it is only necessary to define them at the initial stage of the analysis since they continue to exist even if the CAD model is changed (the groups will adapt to the changes). This is very efficient if there are several iterations of the product development.
	\item The groups are imported to Salomé with the function \texttt{CreateGroups}, this is a simple and time-efficient process that requires very little from the user. Creating a mesh, whether it contains contacts or not, is automated by the functions \texttt{CreateMesh}, \texttt{PartitionShape} and \texttt{CreateSubMesh}, the only manual work is to adjust the mesh settings.
	\item The mesh definitions, and the groups that are required by Code Aster, are automatically created by \texttt{CreateMesh} and \texttt{CreateSubMesh}.
\end{itemize}

% The proposed workflow is presented in Figure~\ref{fig:script_workflow}.

\begin{figure}[t]
	\begin{center}
		\begin{tikzpicture}[node distance=5mm]
			\node[rect] (design)					{Design};
			\node[rect] (clean)	[below=of design] 	{Model clean-up};
			\node[rect] (group) [below=of clean] 	{Create groups in NX};
			\node[rect] (step)	[below=of group]	{Export STEP file};
			\node[rect] (salom)	[below=of step]		{Import to Salomé};
			\node[rect] (cg)	[below=of salom]	{Import groups with \texttt{CreateGroups}};
			\node[rect] (ps)	[below=of cg] 		{If needed use \texttt{PartitionShapes}};
			\node[rect] (cm)	[below=of ps]		{Use \texttt{CreateMesh}};
			\node[rect] (csm)	[below=of cm]	 	{If needed use \texttt{CreateSubMesh}};
			\node[rect] (comm)	[below=of csm]		{Write Code Aster input file};
			\node[rect] (solve)	[below=of comm]		{Solve};
			\node[rect] (vis)	[below=of solve]	{Visualise and analyse results};
			\path (design)	edge[->]	(clean)
				  (clean)	edge[->]	(group)
				  (group)	edge[->]	(step)
				  (step)	edge[->]	(salom)
				  (salom)	edge[->]	(cg)
				  (cg)		edge[->]	(ps)
				  (ps)		edge[->]	(cm)
				  (cm)		edge[->]	(csm)
				  (csm)		edge[->]	(comm)
				  (comm)	edge[->]	(solve)
				  (solve)	edge[->]	(vis);
			\draw [->] (vis.west) -- ++(-.7,0) |- (design.west);
		\end{tikzpicture}
	\end{center}
	\caption{Proposed analysis workflow at Andritz using the functions developed with the Salomé API.}
	\label{fig:script_workflow}
\end{figure}
% subsection workflow_with_salom_script (end)

% section salom_api_script (end)



% chapter analysis (end)